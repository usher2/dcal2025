% !TeX TS-program = xelatex
%\title{Dodecahedron desk calendar}
%%%%%%%%%%%%%%%%%%%%%%%%%%%%%%%%%%%%%%%%%%%%%%%%%%%%%%%%%%%%%%%
%
% Welcome to Overleaf --- just edit your LaTeX on the left,
% and we'll compile it for you on the right. If you give
% someone the link to this page, they can edit at the same
% time. See the help menu above for more info. Enjoy!
%
% Note: you can export the pdf to see the result at full
% resolution.
%
%%%%%%%%%%%%%%%%%%%%%%%%%%%%%%%%%%%%%%%%%%%%%%%%%%%%%%%%%%%%%%%
\documentclass[a4paper]{extarticle}
% Folding + calendar example from the PGF manual.
%
% Author: Till Tantau

\usepackage[rgb,dvipsnames,svgnames]{xcolor}

%% Fonts etc
% \usepackage[T1]{fontenc}
% \usepackage{carlito}
% \usepackage{caladea}
% \usepackage{zi4}

\usepackage{fontspec}
\setmainfont{Carlito}
\setsansfont{Caladea}
\setmonofont{Inconsolata}

\usepackage{url}

%% page layouts
\usepackage[margin=1cm,bottom=2cm,footskip=1.2\baselineskip]{geometry}
\usepackage{fancyhdr}
\fancyhf{}
\renewcommand{\headrule}{}
\rfoot{\normalsize\raisebox{-0.2ex}{\includegraphics[height=1em]{overleaf-logo}} Make your own dodecahedron desk calendar at
\url{https://www.overleaf.com/latex/templates/overleaf-calendar/zhzvgdbvpypg}}
\pagestyle{fancy}

%%%<
\usepackage{verbatim}
% \usepackage[active,tightpage]{preview}
% \PreviewEnvironment{tikzpicture}
% \setlength\PreviewBorder{5pt}%
%%%>

\begin{comment}
:Title: Foldable dodecahedron with Calendar
:Tags: Calendars; Manual
:Author: Till Tantau

An example of the folding library and the folding library, straight from
the manual.
\end{comment}

%%% Uncomment these two lines if you want localisation
%%% of the month names. And you'll then probably want to
%%% Re-define the day-heading list \dhlist below, if you're
%%% using the dayheadings toggle.
% \usepackage[french]{babel}
% \usepackage[french]{translator}



\usepackage{tikz}
\usetikzlibrary{calendar,folding}

%% With further mods by LianTze Lim @ Overleaf (Dec 2016)
\usepackage{OL-calendar-mods}

%%% In case you want to change the weekday headings.
%%% ALWAYS start from 0 and ALWAYS go from left to right
%%% irrespective of whether you are using sundayweek
% \AtBeginDocument{
%   \renewcommand{\dhlist}{0/D,1/L,2/M,3/M,4/J,5/V,6/S}
% }


%%%%% Customisations
\definecolor{OLGreen}{HTML}{4F9C45}
\renewcommand{\SundayColor}{OLGreen}
\renewcommand{\monthcolor}{OLGreen}
\renewcommand{\watermarkfile}{overleaf-logo}

%%%%% SET THE YEAR HERE otherwise the _current_ year is used
\year=2025

%%%%% Uncomment this line to get weeks starting with Sunday
% \toggletrue{sundayweek}

%%%%% Uncomment this line to get weekday headings
\toggletrue{dayheadings}

\begin{document}
    \null\vfill\centering%
    \sffamily\small\noindent
    \begin{tikzpicture}[transform shape]
    \pic [folding line length=9.95em,
        face 1={ \addwatermark\calendar [dates=\the\year-01-01 to \the\year-01-last];},
        face 2={ \addwatermark\calendar [dates=\the\year-02-01 to \the\year-02-last];},
        face 3={ \addwatermark\calendar [dates=\the\year-03-01 to \the\year-03-last];},
        face 4={ \addwatermark\calendar [dates=\the\year-04-01 to \the\year-04-last];},
        face 5={ \addwatermark\calendar [dates=\the\year-05-01 to \the\year-05-last];},
        face 6={ \addwatermark\calendar [dates=\the\year-06-01 to \the\year-06-last];},
        face 7={ \addwatermark\calendar [dates=\the\year-07-01 to \the\year-07-last];},
        face 8={ \addwatermark\calendar [dates=\the\year-08-01 to \the\year-08-last];},
        face 9={ \addwatermark\calendar [dates=\the\year-09-01 to \the\year-09-last];},
        face 10={\addwatermark\calendar [dates=\the\year-10-01 to \the\year-10-last];},
        face 11={\addwatermark\calendar [dates=\the\year-11-01 to \the\year-11-last];},
        face 12={\addwatermark\calendar [dates=\the\year-12-01 to \the\year-12-last];}
    ]{dodecahedron' folding};
    \end{tikzpicture}
    \vfill
\end{document}
