% !TeX TS-program = xelatex
%\title{Dodecahedron desk calendar}
%%%%%%%%%%%%%%%%%%%%%%%%%%%%%%%%%%%%%%%%%%%%%%%%%%%%%%%%%%%%%%%
%
% Welcome to Overleaf --- just edit your LaTeX on the left,
% and we'll compile it for you on the right. If you give
% someone the link to this page, they can edit at the same
% time. See the help menu above for more info. Enjoy!
%
% Note: you can export the pdf to see the result at full
% resolution.
%
%%%%%%%%%%%%%%%%%%%%%%%%%%%%%%%%%%%%%%%%%%%%%%%%%%%%%%%%%%%%%%%
\documentclass[a4paper]{extarticle}
% Folding + calendar example from the PGF manual.
%
% Author: Till Tantau

\usepackage[rgb,dvipsnames,svgnames]{xcolor}

%% Fonts etc
% \usepackage[T1]{fontenc}
% \usepackage{carlito}
% \usepackage{caladea}
% \usepackage{zi4}

\usepackage{fontspec}
\setmainfont{DejaVu Sans}
\setsansfont{Liberation Serif}
\setmonofont{DejaVu Sans Mono}

\usepackage{url}

\usepackage{setspace}
\usepackage[none]{hyphenat}

%% page layouts
\usepackage[margin=1cm,bottom=1.5cm,footskip=\baselineskip]{geometry}
\usepackage{fancyhdr}
\fancyhf{}
\renewcommand{\headrule}{}
\lfoot{\scriptsize Календарь проекта Эшер II: \url{https://usher2.club}, \url{https://t.me/usher2}}
\pagestyle{fancy}

%%%<
\usepackage{verbatim}
% \usepackage[active,tightpage]{preview}
% \PreviewEnvironment{tikzpicture}
% \setlength\PreviewBorder{5pt}%
%%%>

\begin{comment}
:Title: Foldable dodecahedron with Calendar
:Tags: Calendars; Manual
:Author: Till Tantau

An example of the folding library and the folding library, straight from
the manual.
\end{comment}

%%% Uncomment these two lines if you want localisation
%%% of the month names. And you'll then probably want to
%%% Re-define the day-heading list \dhlist below, if you're
%%% using the dayheadings toggle.
\usepackage[russian]{babel}
\usepackage[russian]{translator}

\usepackage{tikz}
\usetikzlibrary{calendar,folding}

%% With further mods by LianTze Lim @ Overleaf (Dec 2016)
\usepackage{OL-calendar-mods}

%%% In case you want to change the weekday headings.
%%% ALWAYS start from 0 and ALWAYS go from left to right
%%% irrespective of whether you are using sundayweek
\AtBeginDocument{
  \renewcommand{\dhlist}{0/Пн,1/Вт,2/Ср,3/Чт,4/Пт,5/Сб,6/Вс}
}


%%%%% Customisations
%\definecolor{OLGreen}{HTML}{4F9C45}
%\renewcommand{\SundayColor}{OLGreen}
%\renewcommand{\monthcolor}{OLGreen}
%\renewcommand{\watermarkfile}{usher_2_logo_bg.png}
\newcommand{\addlogo}{%
        \node[anchor=south,inner sep=0pt,opacity=0.6] at (0, 1.6) {\includegraphics[width=3.3em]{img/usher_2_logo.png}};%
}


\definecolor{RegularColorPallete1}{HTML}{4B2E07}
\definecolor{SundayColorPallete1}{HTML}{CC6D1A}
\definecolor{MonthColorPallete1}{HTML}{E67E22}
\definecolor{BlockQuoteColorPallete1}{HTML}{6D481F}

\definecolor{RegularColorPallete2}{HTML}{5C3D10}
\definecolor{SundayColorPallete2}{HTML}{FF7F27}
\definecolor{MonthColorPallete2}{HTML}{D35400}
\definecolor{BlockQuoteColorPallete2}{HTML}{7C592C}

\definecolor{RegularColorPallete3}{HTML}{4F2817}
\definecolor{SundayColorPallete3}{HTML}{F98948}
\definecolor{MonthColorPallete3}{HTML}{FD7326}
\definecolor{BlockQuoteColorPallete3}{HTML}{7B4833}

\definecolor{RegularColorPallete4}{HTML}{3D2B1F}
\definecolor{SundayColorPallete4}{HTML}{D87F33}
\definecolor{MonthColorPallete4}{HTML}{F39C12}
\definecolor{BlockQuoteColorPallete4}{HTML}{5F4939}


\newcommand{\RegularColor}{RegularColorPallete2}
\color{RegularColorPallete2}

\renewcommand{\SundayColor}{SundayColorPallete2}
\renewcommand{\monthcolor}{MonthColorPallete2}

\colorlet{BlockQuoteColor}{BlockQuoteColorPallete2}

%%%%% SET THE YEAR HERE otherwise the _current_ year is used
\year=2025

%%%%% Uncomment this line to get weeks starting with Sunday
% \toggletrue{sundayweek}

%%%%% Uncomment this line to get weekday headings
\toggletrue{dayheadings}

\begin{document}
    \hyphenpenalty=10000%
    \exhyphenpenalty=10000%
    \null\vfill\centering%
    \sffamily\small\noindent%
    \begin{tikzpicture}[transform shape]
    \pic [folding line length=11em,
        face 1={\addlogo\calendar [dates=\the\year-01-01 to \the\year-01-last,
        if={(equals=\the\year-01-01) [\SundayColor]},
        if={(equals=\the\year-01-02) [\SundayColor]},
        if={(equals=\the\year-01-03) [\SundayColor]},
        if={(equals=\the\year-01-06) [\SundayColor]},
        if={(equals=\the\year-01-07) [\SundayColor]},
        if={(equals=\the\year-01-08) [\SundayColor]}];%
        \node[align=center, text width=11em,inner sep=0, text=BlockQuoteColor] at (0, -1.9) {\begin{spacing}{0.8}\scriptsize\textit{Торжественно клянусь, что замышляю шалость, и только шалость!}\end{spacing}};},
        face 2={\addlogo\calendar [dates=\the\year-02-01 to \the\year-02-last];
        \node[align=center, text width=11em,inner sep=0, text=BlockQuoteColor] at (0, -1.9) {\begin{spacing}{0.8}\scriptsize\textit{\textbf{X-Ray (Project X)}: экосистема протоколов и технологий. Популярные комбинации - XTLS-Reality и XHTTP}\end{spacing}};},
        face 3={\addlogo\calendar [dates=\the\year-03-01 to \the\year-03-last];
        \node[align=center, text width=11em,inner sep=0, text=BlockQuoteColor] at (0, -2.1) {\begin{spacing}{0.8}\scriptsize\textit{\textbf{Wireguard}: очень простой протокол VPN, легко обнаруживается}\end{spacing}};},
        face 4={\addlogo\calendar [dates=\the\year-04-01 to \the\year-04-last];
        \node[align=center, text width=11em,inner sep=0, text=BlockQuoteColor] at (0, -1.9) {\begin{spacing}{0.8}\scriptsize\textit{\textbf{Amnezia Wireguard}: изменённый для усложнения обнаружения Wireguard. Разработка AmneziaVPN}\end{spacing}};},
        face 5={\addlogo\calendar [dates=\the\year-05-01 to \the\year-05-last,
        if={(equals=\the\year-05-01) [\SundayColor]},
        if={(equals=\the\year-05-02) [\SundayColor]},
        if={(equals=\the\year-05-08) [\SundayColor]},
        if={(equals=\the\year-05-09) [\SundayColor]}];%
        \node[align=center, text width=11em,inner sep=0, text=BlockQuoteColor] at (0, -1.9) {\begin{spacing}{0.8}\scriptsize\textit{\textbf{VPN}: частная виртуальная сеть. Для соединения сетей, защиты соединения, а теперь и обхода блокировок}\end{spacing}};},
        face 6={\addlogo\calendar [dates=\the\year-06-01 to \the\year-06-last,
        if={(equals=\the\year-06-12) [\SundayColor]},
        if={(equals=\the\year-06-13) [\SundayColor]}];%
        \node[align=center, text width=11em,inner sep=0, text=BlockQuoteColor] at (0, -2.1) {\begin{spacing}{0.8}\scriptsize\textit{\textbf{ShadowSocks}: развитие прокси Socks5 с шифрованием}\end{spacing}};},
        face 7={\addlogo\calendar [dates=\the\year-07-01 to \the\year-07-last];
        \node[align=center, text width=11em,inner sep=0, text=BlockQuoteColor] at (0, -1.9) {\begin{spacing}{0.8}\scriptsize\textit{\textbf{Tor}: (луковый маршрутизатор) - позволяет относительно анонимно пользоваться сетью и обходтить блокировки}\end{spacing}};},
        face 8={\addlogo\calendar [dates=\the\year-08-01 to \the\year-08-last];
        \node[align=center, text width=11em,inner sep=0, text=BlockQuoteColor] at (0, -1.9) {\begin{spacing}{0.8}\scriptsize\textit{\textbf{Ceno}: браузер для обхода цензуры с использованием торрент-протокола. Проект eQualit.ie}\end{spacing}};},
        face 9={\addlogo\calendar [dates=\the\year-09-01 to \the\year-09-last];
        \node[align=center, text width=11em,inner sep=0, text=BlockQuoteColor] at (0, -1.9) {\begin{spacing}{0.8}\scriptsize\textit{\textbf{Эшер II}: проект существующий с целью курощения, низведения и дуракаваляния Роскомнадзора}\end{spacing}};},
        face 10={\addlogo\calendar [dates=\the\year-10-01 to \the\year-10-last];
        \node[align=center, text width=11em,inner sep=0, text=BlockQuoteColor] at (0, -1.9) {\begin{spacing}{0.8}\scriptsize\textit{\textbf{3x-UI}, \textbf{AmneziaVPN}, \textbf{VPN Generator}, \textbf{Outline}, \textbf{Ceno}, \textbf{Tor}: от врагов}\end{spacing}};},
        face 11={\addlogo\calendar [dates=\the\year-11-01 to \the\year-11-last,
        if={(equals=\the\year-11-1) [\RegularColor]},
        if={(equals=\the\year-11-3) [\SundayColor]},
        if={(equals=\the\year-11-4) [\SundayColor]}];%
        \node[align=center, text width=11em,inner sep=0, text=BlockQuoteColor] at (0, -1.9) {\begin{spacing}{0.8}\scriptsize\textit{\textbf{Outline}: система автоматической установки и клиент VPN на основе ShadowSocksAEAD}\end{spacing}};},
        face 12={\addlogo\calendar [dates=\the\year-12-01 to \the\year-12-last,
        if={(equals=\the\year-12-31) [\SundayColor]}];%
        \node[align=center, text width=11em,inner sep=0, text=BlockQuoteColor] at (0, -1.9) {\begin{spacing}{0.8}\scriptsize\textit{Шалость удалась!}\end{spacing}};}
    ]{dodecahedron' folding};
    \end{tikzpicture}
    \vfill
\end{document}
